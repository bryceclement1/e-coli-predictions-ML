
\section{Broader Impacts}
\label{sec:impacts}

Our models predict the \textit{E. coli} concentration in water bodies from the Great Lakes region in Ohio and Pennsylvania. Based on data points related to such as water and weather conditions, we produced models that have similar or even better performances than the published model, indicating the possibility of generating better prediction using ours. However, a few concerns still remain in our mind at the end of this project.\\
The first concern is the generality of our model. Since the data was collected near two beaches that are considered as on-shore or near-shore areas, the models are limited by the data because it could only be applicable to similar environments. Furthermore, the prediction should not be reliable for predicting water quality in regions that are not similar as the Great Lakes region, which has a unique and complex system of environmental components. We believe that although the results of our models look good on the paper, the practicality of actual real-life use should be concerned before application.\\
The second is about the general data-driven prediction of water quality. We believe that it should be cautious to take experimental models into real-life applications. Using data-driven model for water quality prediction could potentially cause major public health issues if done wrong. As mentioned above, the limitation of both our models and the published one is obvious and should be used with caution when used in real life. One way to make the models more applicable is to limit the scope: for example, concentrating on predicting water quality in agricultural pond water \cite{Stocker}. By limiting the prediction target, models could be more reliable when used for the specific scenario when data collection were standardized ideally. 